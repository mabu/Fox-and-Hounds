\documentclass[a4paper]{article}

\title{Q-learning for Fox and Hounds Board Game}
\author{Javad Bakhshi \and Martynas Budriunas}
\date{2012-05-??}

\begin{document}

\maketitle

\begin{abstract}
Lorem ipsum dolor sit amet, consectetur adipiscing elit. Suspendisse
pellentesque enim eget nunc dapibus id imperdiet sapien dictum. Fusce nulla
sem, fringilla in fermentum ac, pretium eget purus. Sed vel dui diam, ac mattis
nisl. Suspendisse risus metus, ullamcorper at ultrices ac, mollis quis urna.
Nunc libero felis, consectetur eget elementum id, malesuada eu purus. Cras
commodo tincidunt purus, sit amet imperdiet ipsum vestibulum ut. Aenean nec
sapien eget lectus feugiat interdum. In malesuada lorem sed ligula auctor a
dignissim sem auctor. Sed et nisi sit amet turpis tincidunt pulvinar ut non
ante. Aliquam vitae urna dui. Aenean in nisl. 
\end{abstract}

\section{Introduction}
Fox and Hounds is a simple two-player board game. Traditionally it is played on
a chess board. One player controls the fox and the other controls four hounds.
Players have different goals. The fox has to reach the other side of the board,
and the hounds have to surround the fox so that it cannot move.

Initially the hounds are placed on all black squares in one row on the edge of
the board. The fox can choose a starting position on any of black squares in the
row on the opposite edge of the board. The pieces can only move to unoccupied
adjacent black squares. Additionally, the hounds can only move forward. The
players take alternative turns with the fox moving first. On each turn only one
of the hounds can be moved.

Fox and Hounds is an easy game aimed at young children. A human player can
easily find an optimal strategy for the hounds which guarantee victory. In this
project we try to investigate how good a computer can be in finding that
solution. To do that we have implemented learning systems for both players
based on Q-learning.

\section{Q-learning}

\section{Implementation}

\section{Expirements}

\section{Results}

\section{Conclusions and Future Work}

\begin{thebibliography}{9}
\end{thebibliography}

\end{document}

